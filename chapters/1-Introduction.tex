
\chapter{Introduction}
\section{Prerequisites and Basic Formulae}
\subsection{Mathematical Notation}
\begin{itemize}
    \item $\equiv$ - is used a lot of example calculations, this symbols means that this is the definition of the thing. 
    \item
\end{itemize}

\subsection{General Relativity}
\subsubsection{Metrics}
\begin{itemize}
    \item $\eta_{\mu\nu}$ - is the \textbf{Minkowski metric}. We use the \textit{mostly-plus} signature, $\eta = \text{diag}(-1, 1, 1, 1)$, which we use.
    \item $g_{\mu\nu}$ - is used for the metric of a Lorentzian spacetime.
    \item $\partial_{\mu} = \frac{\partial}{\partial x^{\mu}}$ - denotes a \textbf{directional derivative} or the usual derivative. It can be thought of as the higher-dimensional analogue of the \textit{gradient operator} in vector calculus. Often we will use shorthand such as $\omega_{\mu,\rho}=\partial_{\rho}\omega_{\mu}$. So something more complicated like, $\partial_{\sigma}\partial_{\rho}T^{\xi\ \chi}_{\ \tau} = T^{\xi\ \chi}_{\ \tau \ ,\sigma\rho}$.
    \item Christoffel Connection ($\Gamma^{\alpha}_{\ \mu\nu}$) are a system of numbers which are constructed from the derivatives of the metric,
    \begin{equation}
    \Gamma^{\alpha}_{\ \mu\nu}\equiv\frac{1}{2}g^{\alpha\beta}\left(\partial_{\mu}g_{\nu\beta}+\partial_{\nu}g_{\mu\beta} - \partial_{\beta}g_{\mu\nu}\right).
    \end{equation}
    The Christoffel Connection allows us to construct a \textit{covariant derivative}, as well as other covariant properties of manifolds. 
    \item $\nabla_{\mu}$ -  will denote the \textit{covariant derivative} of a tensor field. For (pseudo)-Riemannian manifolds, the covariant derivative on a vector field is written, 
    \begin{equation}
        \nabla_{\mu}V^{\rho} = \partial_{\mu}V^{\rho} + \Gamma^{\rho}_{\ \mu \sigma}V^{\sigma}
    \end{equation}
    On scalar fields, $\lambda(x)$, the covariant derivative takes on a particularly attractive form, $\nabla_{\mu}\lambda(x) = \partial_{\mu}\lambda(x)$.
    On one-form (covector) fields, the Christoffel term has a negative sign,
    \begin{equation}
        \nabla_{\mu}\omega_{\nu} = \partial_{\mu}\omega_{\nu} - \Gamma^{\kappa}_{\ \mu \nu}\omega_{\kappa}
    \end{equation}
    Generally, on $(n,m)$-tensors, the covariant derivative generates and $(n,m+1)$-tensor:
    \begin{equation}
    \begin{split}
        \nabla_{\rho}T^{\mu_{1}\dots\mu_{n}}_{\nu_{1}\dots\nu_{m}} &= \partial_{\rho}T^{\mu_{1}\dots\mu_{n}}_{\nu_{1}\dots\nu_{m}} + \Gamma_{\ \ \rho\kappa}^{\mu_{1}}T^{\kappa\dots\mu_{n}}_{\nu_{1}\dots\nu_{m}}+\dots + \Gamma_{\ \ \rho\kappa}^{\mu_{n}}T^{\mu_{1}\dots\kappa}_{\nu_{1}\dots\nu_{m}}\\
        &\ \ \ \ \ \ \ \ \ \ \ \ \ \ \ \ - \Gamma_{\ \ \rho\nu_{1}}^{\kappa}T^{\mu_{1}\dots\mu_n}_{\kappa\dots\nu_{m}}-\dots - \Gamma_{\ \ \rho\nu_m}^{\kappa}T^{\mu_{1}\dots\mu_n}_{\nu_1\dots\kappa}.
    \end{split}
    \end{equation}
    The rule to remember for the Christoffel terms is that $\Gamma$ sums over every index separately, with a $+$ sign for \textit{upstairs} indices, and a $-$ sign for \textit{downstairs} indices. However, there is a unique tensor, which the covariant derivative has a special effect on: 
    \begin{equation}\label{metric-compatible}
        \nabla_{\alpha}g_{\mu\nu} \equiv 0
    \end{equation}
    This is actually not a huge surprise as it comes from the definition of the $\Gamma^{\rho}_{\ \mu\nu}$. (\ref{metric-compatible}) is the \textbf{\textit{condition of metric-compatibility}}.
\end{itemize}

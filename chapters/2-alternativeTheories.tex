\chapter{Alternative Theories of Gravity}
\section{Scalar-Tensor Theories}
\textbf{(Heavily borrowed from Carroll Chapter 4)} Despite the continual success of General Relativity in describing gravitational interactions, it is always possible that the next experiment we perform could show considerable deviations and reveal effects that we cannot describe using it. In principle, there is an infinite range of modifications that could be made to GR, however, there are a few that we direct special attention to. For later comparisons and a reminder the usual Einstein-Hilbert action in $(1+3)$-dimensional Lorentzian spacetime, $\mathcal{M}$ is,
\begin{equation}
    S_{R} = \frac{1}{16\pi G}\int_{\mathcal{M}} d^4 x\sqrt{-g}R
\end{equation}
where, $R$ is the Ricci scalar, $G$ is the Gravitational Constant and $g = \det(g_{\mu\nu})$ is the metric determinant of our spacetime. This leads to the usual Einstein Field Equations in the presence of matter, 
\begin{equation}\label{efes}
    G_{\mu\nu} = 8\pi GT_{\mu\nu}
\end{equation}
A popular set of theories is known as \textbf{scalar-tensor theories} since they contain a metric tensor, $g_{\mu\nu}$, and a scalar field $\varphi$ \citep{carrolGravity}. The action of these theories can be written as the following sum, 

\begin{equation}
    S = S_{fR} + S_{\varphi} + S_M,
\end{equation}
where,
\begin{eqnarray}\label{st-actions}
    S_{fR} &=& \int d^4x\sqrt{-g}f(\varphi)R,\\
    S_{\varphi} &=& \int d^4x\sqrt{-g}\left[-\frac{1}{2}\mathfrak{h}(\varphi)g^{\mu\nu}\partial_{\mu}\varphi\partial_{\nu}\varphi - U(\varphi)\right],
\end{eqnarray}
and 
\begin{equation}\label{matter-action}
    S_M = \int d^4x\sqrt{-g} \hat{\mathcal{L}}_M(g_{\mu\nu}, \psi_i).
\end{equation}
The functions $f(\varphi)$, $\mathfrak{h}(\varphi)$ and $U(\varphi)$ define the theory we are interested in and the matter Lagrangian $\hat{\mathcal{L}}_M$ is a function of the metric and the matter fields $\psi_i$ but not of $\varphi$. By the usual variations of the inverse metric $g^{\mu\nu} \rightarrow g^{\mu\nu}  + \delta g^{\mu\nu} $, we can obtain the Field Equations of this system. Variations with respect to the inverse metric can be used to obtain the variations with respect to the metric using the following conversion,

\begin{eqnarray}
    \delta(g^{\mu\sigma}g_{\nu\sigma}) &=& \delta g^{\mu\sigma}g_{\nu\sigma} + g^{\mu\sigma}\delta g_{\nu\sigma}= 0 \nonumber\\
    \delta g_{\mu\nu} &=& -g_{\mu\rho}g_{\nu\sigma}\delta g^{\rho\sigma}
\end{eqnarray}
Focusing on the gravitational contributions for the moment, from the Leibnitz product rule, we have,
\begin{equation}
    \delta S_{fR} = \delta S_1 + \delta S_2 + \delta S_3\\
\end{equation}

where,
\begin{eqnarray}
    \delta S_1 = \int d^4 x \sqrt{-g} f(\varphi) g^{\mu\nu}\delta R_{\mu\nu}\\
    \delta S_2 = \int d^4 x \sqrt{-g} f(\varphi) R_{\mu\nu} \delta g^{\mu\nu}\\
    \delta S_3 = \int d^4 x \sqrt{-g} f(\varphi) R\delta\sqrt{-g}
\end{eqnarray}
We see that $\delta S_2$ is in the form that we require, however, $\delta S_1$ and $\delta S_3$ require some extra care. Starting with $\delta S_1$, we use the Riemann tensor, 
\begin{equation}
    R^{\rho}_{\ \mu \lambda \nu} = \partial_{\lambda}\Gamma^{\rho}_{\ \nu \mu} + \Gamma^{\rho}_{\ \lambda \sigma} \Gamma^{\sigma}_{\ \nu\mu} - \partial_{\nu}\Gamma^{\rho}_{\ \lambda \mu} - \Gamma^{\rho}_{\ \nu \sigma} \Gamma^{\sigma}_{\ \lambda\mu}
\end{equation}
Following steps...
The variations of the metric lead to a gnarly integral, 
\begin{equation}
    \delta S_{fR} = \int d^4x\sqrt{-g}f(\varphi)\left[\left(R_{\mu\nu}- \frac{1}{2} Rg_{\mu\nu}\right)\delta g^{\mu\nu} + \nabla_{\sigma}\nabla^{\sigma}(g_{\mu\nu}\delta g^{\mu\nu}) - \nabla_{\mu}\nabla_{\nu}(\delta g^{\mu\nu})\right]
\end{equation}
For the usual Einstein-Hilbert action, $f(\varphi)$ is a constant which means the last two terms are total derivatives, which we can integrate by parts. This will furnish the action with terms involving derivatives of $f(\varphi)$,
\begin{equation}
    \delta S_{fR} = \int d^4x\sqrt{-g}\left[G_{\mu\nu}f(\varphi) + g_{\mu\nu}\Box f(\varphi)  - \nabla_{\mu}\nabla_{\nu}f(\varphi)\right]\delta g^{\mu\nu}
\end{equation}
with the usual $G_{\mu\nu} = R_{\mu\nu}- \frac{1}{2} Rg_{\mu\nu}$. The Field equations are then given by, 
\begin{equation}\label{modified-efes}
    G_{\mu\nu} = \frac{1}{f(\varphi)}\left(\frac{1}{2}T_{\mu\nu}^{(M)} + \frac{1}{2}T_{\mu\nu}^{(\varphi)} + \nabla_{\mu}\nabla_{\nu}f - g_{\mu\nu}\Box f\right)
\end{equation}
wherein the energy-momentum tensors are defined in the standard sense in GR,
\begin{equation}
T^{(I)}_{\mu\nu} = -\frac{2}{\sqrt{-g}}\frac{\delta S_{I}}{\delta g^{\mu\nu}},
\end{equation}
where the index $I$ denotes the matter or scalar parts of the actions. In particular, 
\begin{equation}
    T_{\mu\nu}^{(\varphi)} = \mathfrak{h}(\varphi)\nabla_{\mu}\varphi\nabla_{\nu}\varphi - g_{\mu\nu}\left[\frac{1}{2}\mathfrak{h}(\varphi)g^{\rho\sigma}\nabla_{\rho}\varphi\nabla_{\sigma}\varphi + U(\varphi)\right]
\end{equation}
Looking closely at (\ref{modified-efes}) and (\ref{efes}), we can identify $f(\varphi) = 1/16\pi G$, in the Einstein-Hilbert case. This identification gives us the usual GR and so is nothing new. However, this identifies a gravity-like theory, but with a strength that varies with the value of the scalar $\varphi$ at different spacetime positions. This places a bound on how large the scalar field amplitude can be on cosmological scale and the scale of the Solar System (because it would be obviously observable... and we can't observe it). We can determine the equations of motion for $\varphi$, 
\begin{equation}
    \mathfrak{h}(\varphi)\Box\varphi + \frac{1}{2}\mathfrak{h}'(\varphi)g^{\mu\nu}(\nabla_{\mu}\varphi)(\nabla_{\nu}\varphi) - U'(\varphi) + f'(\varphi)R = 0
\end{equation}
where the prime $'$ denotes differentiation with respect to $\varphi$. If $\mathfrak{h}(\varphi) = 1$ the conventional scalar field equations are obtained, 
\begin{equation}
    \Box\varphi + f'(\varphi)R= U'(\varphi)
\end{equation}
but with this curious Ricci scalar coupling. From the previous discussion, we can see that with little variations in $f(\varphi)$ this new term can be dropped to obtain the usual dynamics of $\varphi$. Another mechanism to lessen this effect is to choose a potential with a minimum and ensuring that $\varphi$ cannot deviate from this minimum without a sufficient energy input. In other words, a very large mass for $\varphi$. Take a potential with $U(0) = U'(0) = 0$, 

\begin{equation}
    U(\varphi) \ \approx\  \frac{1}{2}m_{\varphi}^2\varphi^2 + ...
\end{equation}
where we identify $m_{\varphi}^2 = U''(0)>0$ and $m_{\varphi}^2\gg 1$. Or we could choose $f$ and $\mathfrak{h}$ so that large changes in $\varphi$ give rise to relatively small changes in the effective changes in $G$. 

\section{Brans-Dicke Theory}
A famous example of a scalar-tensor theory is \textbf{Brans-Dicke theory}, and corresponds to the following choices,
\begin{eqnarray}
    f(\varphi) = \frac{\varphi}{16\pi},\ \ \ \ \mathfrak{h}(\varphi) = \frac{\lambda}{8\pi\varphi},\ \ \ \ U(\varphi) = 0, 
\end{eqnarray}
where $\xi$ is a coupling constant. The scalar-tensor action is written as, 
\begin{equation}
    S_{\text{BD}} = \int d^4x\frac{\sqrt{-g}}{16\pi}\left[R\varphi -\lambda g^{\mu\nu}\frac{(\partial_{\mu}\varphi)(\partial_{\nu}\varphi)}{\varphi}\right].
\end{equation}
Brans-Dicke theory chooses a massless scalar. But in the limit $\lambda\rightarrow \infty$ the field becomes non-dynamical and ordinary GR is recovered. Current tests in the solar system imply $\lambda>500$, of if there is such a scalar field it must couple only weakly to $R$. 

\subsection{Using Conformal Transformations}
One way to deal with scalar-tensor theories is to perform a conformal transformation to make the connection to gravity more apparent. We choose the conformal metric, $\tilde{g}_{\mu\nu}$, 
\begin{equation}\label{conformal}
    \tilde{g}_{\mu\nu} = 16\pi\tilde{G}f(\varphi)g_{\mu\nu}
\end{equation}
where $\tilde{G}$ is the Gravitational constant in the conformal frame. The connection coefficients in this new frame ($\tilde{\Gamma}^{\mu}_{\ \rho\sigma}$) are, 
\begin{equation}
    \tilde{\Gamma}^{\rho}_{\ \mu\nu} = \frac{1}{2}\tilde{g}^{\rho\sigma}\left(\tilde{g}_{\mu\sigma, \nu} + \tilde{g}_{\nu\sigma, \mu} - \tilde{g}_{\mu\nu, \sigma}  \right)
\end{equation}
Substituting the conformal metric (\ref{conformal}) in terms of the usual metric give, 
\begin{eqnarray}
    \tilde{\Gamma}^{\rho}_{\ \mu\nu} &=& \Gamma^{\rho}_{\ \mu\nu} + \frac{1}{2}f(\varphi)^{-1}\left((\partial_{\nu}f)\delta^{\rho}_{\mu} + (\partial_{\mu}f)\delta^{\rho}_{\nu} -g^{\rho\sigma}g_{\mu\nu}(\partial_{\sigma}f)\right)\\
    &=& \Gamma^{\rho}_{\ \mu\nu} + C^{\rho}_{\ \mu \nu},
\end{eqnarray}
where $C^{\rho}_{\ \mu \nu}$ is the difference between connections and so is a tensor (see Appendix \ref{appendixConformalDynamics}). Using the chain rule, $\partial_{\mu}f(\varphi) = (\partial f/\partial \varphi)\partial_{\mu}\varphi$, we can write, 
\begin{equation}
    \tilde{\Gamma}^{\rho}_{\ \mu\nu} = \Gamma^{\rho}_{\ \mu\nu} + \frac{1}{2}\frac{\partial \ln[f(\varphi)]}{\partial \varphi}\left[(\partial_{\nu}\varphi)\delta^{\rho}_{\mu} + (\partial_{\mu}\varphi)\delta^{\rho}_{\nu} -g^{\rho\sigma}g_{\mu\nu}(\partial_{\sigma}\varphi)\right]
\end{equation}
This connection defines a new covariant derivative on a vector field, $V = V^{\mu}(x)\partial_{\mu}$, 
\begin{eqnarray*}
\tilde{\nabla}_{\mu}V^{\rho} &=& \partial_{\mu}V^{\rho} + \tilde{\Gamma}^{\rho}_{\ \mu\nu}V^{\nu},\\
&=& \partial_{\mu}V^{\rho} + \Gamma^{\rho}_{\ \mu\nu}V^{\nu} + \frac{1}{2}\frac{\partial \ln[f(\varphi)]}{\partial \varphi}\left[(V^{\nu}\partial_{\nu}\varphi)\delta^{\rho}_{\mu} + (\partial_{\mu}\varphi)V^{\rho} -g^{\rho\sigma}(\partial_{\sigma}\varphi)V_{\mu}\right]
\end{eqnarray*}
and on a covector field, $\omega = \omega_{\mu} (x)dx^{\mu}$, 
\begin{eqnarray*}
    \tilde{\nabla}_{\mu}\omega_{\nu} &=& \partial_{\mu}\omega_{\nu} - \tilde{\Gamma}^{\rho}_{\ \mu\nu}\omega_{\rho},\\
    &=& \partial_{\mu}\omega_{\nu} - \Gamma^{\rho}_{\ \mu\nu}\omega_{\rho} - \frac{1}{2}\frac{\partial \ln[f(\varphi)]}{\partial \varphi}\left[(\partial_{\nu}\varphi)\omega_{\mu} + (\partial_{\mu}\varphi)\omega_{\nu} -g_{\mu\nu}(\partial_{\sigma}\varphi)\omega^{\sigma}\right]
\end{eqnarray*}
which generalises to higher-order tensors in the usual manner. Continuing with the general calculations of the Riemann and Ricci tensors, we obtain the following action in terms of the conformal Ricci Scalar,
\begin{equation}
    S_{fR}= \int d^4x\ \frac{\sqrt{-g}}{16\pi\tilde{G}}\left[\tilde{R} + \frac{3}{2}\tilde{g}^{\rho\sigma}f^{-2}\left(\frac{df}{d\varphi}\right)^2(\tilde{\nabla}_{\rho}\varphi)(\tilde{\nabla}_{\sigma}\varphi)\right]
\end{equation}
where we have integrated by parts and discarded surface terms. In the conformal frame, therefore, the curvature scalar appears by itself, which looks a lot like GR. For this reason, this frame is called the \textbf{Einstein frame}, since the Einstein equations for the conformal metric $\tilde{g}_{\mu\nu}$ take on their conventional form. The original frame with metric $g_{\mu\nu}$ is called the \textbf{Jordan frame}. 

If we make the choice,
\begin{equation}
    f(\varphi) = e^{\varphi/\sqrt{3}}, \ \ \mathfrak{h}(\varphi) = U(\varphi) = 0 
\end{equation}
which is a specific choice of $f(\varphi)$ and turns off the $S_{\varphi}$ contribution from the pure scalar. With these choices, in the Einstein frame, we see a conventional kinetic term for the scalar field,
\begin{equation}
    S = \int d^{4}x \sqrt{-\tilde{g}}(16\pi\tilde{G})^{-1}\left[\tilde{R} - \frac{1}{2}\tilde{\nabla}_{\rho}\varphi\tilde{\nabla}^{\rho}\varphi\right] + S_{M}.
\end{equation}
Even without an explicit kinetic term in the Jordan frame, we see a kinetic term generated in the Einstein frame. 
\bibliographystyle{numeric}
\bibliography{chapters/bibliographies/2-alternativeTheories}
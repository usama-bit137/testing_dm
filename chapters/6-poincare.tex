\chapter{Particle Physics and Symmetries}
\section{The Poincar\'e Group}
We have argues that in a quantum mechanical theory with Hilbert space $\mathcal{H}$ that physical symmetries correspond to unitary representations of a group $G$ on $\mathcal{H}$ (unless the symmetry involves times-reversal). Since symmetries play a vital role in particle physics we will try and identify:
\begin{itemize}
    \item which groups $G$ are relevant to particle physics.
    \item what representations appear.
\end{itemize}
There are three types of symmeties:
\begin{itemize}
    \item \textit{spacetime symmetries}: Poincar\'e group
    \item \textit{gauge symmetries} describe interactions
    \begin{itemize}
        \item EM
        \item strong force
        \item weak force
    \end{itemize}
    \item global symmetries
    \begin{itemize}
        \item exact: lepton number, baryon number, \dots
        \item approximate: isospin, quark model, \dots
    \end{itemize}
\end{itemize}

\subsection{Representations of the Poincar\'e group}
The Poincar\'e is a matrix group 
\begin{equation}
    ISO(3,1) = \left\{\begin{pmatrix}
        \Lambda & a \\
        0 & 1
    \end{pmatrix}\in GL(5,\mathbb{R}): \Lambda \in O(3,1)\right\}
\end{equation}
acting $v = \begin{pmatrix}
    x\\
    1
\end{pmatrix}$
\begin{equation}
    x^{\mu}\mapsto \Lambda^{\mu}_{\ \nu}x^{\nu} + a^{\mu}
\end{equation}

Then Lorentz group is the subgroup with $a=0$. The Lorentz group has two subgroups
\begin{itemize}
    \item \textit{orthochronous}: $O^{+}(3,1)$,
    \begin{equation}
    \begin{split}
        \text{\textit{future timelike}}&\longrightarrow \text{\textit{future timelike (no time reversal)}}\\
        \Lambda^{0}_{\ 0}&\geq 1
    \end{split}
    \end{equation}
    \item proper: $SO(3,1)$: $\det\Lambda = 1$ (no reflections)
\end{itemize}

It is natural to consider the intersection of these Lorentz subgroups: 
\begin{equation}
    \begin{split}
        SO^+(3,1): SO(3,1)\cap O^{+}(3,1) \text{ \textit{ proper and orthochronous} Lorentz group}\\
        ISO^+(3,1): \text{ \textit{ proper and orthochronous} Poincar\'e group}
    \end{split}
\end{equation}
The other \textit{bits} of the Poincar\'e group can be reached with
\begin{itemize}
    \item time reversal $\Lambda = \text{diag}(-1,1,1,1)$
    \item parity: $\Lambda = \text{diag}(1,-1,-1,-1)$
\end{itemize}
or the combinations of the above.

A theorem by Wigner classified the unitary representations of $ISO^+(3,1)$ with \textit{non-negative energies} and definite \textit{mass}. We can roughly prove this focusing on the essence of the proof, rather than the mathematical details. Note that $ISO^+(3,1)$ is \textit{non-compact} (boosts and translations are not bounded) and so: 
\begin{itemize}
    \item unitary representations are $\infty$-dimensional $S(\Lambda,a)$
\end{itemize}
(The Lorentz group is also non-compact thus has no finite dimensional unitary representations).

Wigner starts with the \textit{translation subgroup}: 
\begin{equation}
    x^{\mu}\mapsto x^{\mu}
\end{equation}
$G = \mathbb{R}^4$ which is four copies of Abelian group. We know that the Abelian irreps are one-dimensional. For $G = \mathbb{R}$ under addition, irrep is labelled by the \textit{charge}, 
\begin{equation}
    \rho_{q} = e^{-iqa}
\end{equation}
where $q\in\mathbb{R}$ which is no faithful. For $G = \mathbb{R}^4$ we have four charges, 

\begin{equation}
    \rho_{p_n} = e^{-ip\cdot a}
\end{equation}
where $p\cdot a = p_{\mu}a^{\mu}$ and $p^{\mu} = (E, \vec{p})$. In terms of the Hilbert space, 

\begin{itemize}
    \item Hilbert space: each irrep is one-dimensional, $\ket{p^{\mu}}$ and labels the basis, 
\begin{equation}
    \mathcal{H}_{p^{\mu}} = \{\lambda\ket{p^{\mu}}:\lambda\in \mathbb{C}\}
\end{equation}
\item unitary rep: 
\begin{equation}
    S(1,a)\ket{p^{\mu}} = e^{-ip\cdot a}\ket{p^{\mu}}
\end{equation}
\end{itemize}
and \textit{non-negative} energy means $E>0$. \textit{Momentum} of the state $p^{\mu}$ just labels \textit{translation irrep}.

Next we consider the Lorentz group part. For this rep, we require
\begin{equation}
    S(\Lambda, 0)S(1,a)S(\Lambda, 0)^{-1} = S(\Lambda\Lambda^{-1}, \Lambda a) = S(0, \Lambda a)
\end{equation}
Then we can form a \textit{unitary rep} of the Poincar\'e group by

\begin{equation}
    \begin{split}
        S(\Lambda, 0)&S(1,a)S(\Lambda, 0)^{-1}\ket{\Lambda p} \\
        &= S(\Lambda, 0)S(1, a)\ket{p} = e^{-ip\cdot a}S(\Lambda,0)\ket{p}\\
        &=e^{ip\cdot a}\ket{\Lambda p}\\
        &=S(1,\Lambda a)\ket{\Lambda p} = e^{-i\Lambda p\cdot \Lambda a}\ket{\Lambda p}
    \end{split}
\end{equation}
Actually we don't have to take all states: only the ones with fixed $p^2$

\begin{equation}
    \mathcal{H} = \bigoplus_{p^{\mu}: \ p^2 = m^2}\mathcal{H}_p
\end{equation}
where in general, 
\begin{equation}
    S(\Lambda, a)\ket{p} = S(1, a)S(\Lambda, 0)\ket{p} = e^{-i\Lambda p\cdot a}\ket{\Lambda p}
\end{equation}
In the simplest case: $p^{\mu} =0$ 
\begin{equation}
    \mathcal{H} = \{\lambda\ket{0}: \lambda\in\mathbb{C}\}
\end{equation}
$S(\Lambda, a)\ket{0} = \ket{0}$ which transforms under the trivial representation, this is the \textit{ground state} of QFT. But for $m^2\geq 0$
\begin{equation}
\begin{split}
    \mathcal{H} = \left\{\ket{\psi} &= \int\frac{d^3p}{(2\pi)^3}\frac{1}{2E_{\vec{p}}}\psi(\vec{p})\ket{p}: p^2=m^2\right\} \\
    &= \text{one-particle Hilbert space of QFT (scalar)}
\end{split}
\end{equation}
Now note that we have, 
\begin{equation}
    \mathcal{H} = \{\text{positive energy solutions of the Klein-Gordon equation}\}
\end{equation}
So, one way to view the $\infty$-dimensional irrep of the Poincar\'e group: \textbf{field representation}.
This construction only gives spin-$0$ rep. To see the other spins, consider 
\begin{enumerate}
    \item $p^2 = m^2 > 0$
    \item $p^2=0$
\end{enumerate}
We are able to perform boosts in case 1 so we move to the intertial frame of our particle, $p^{\mu} = (E,0,0,0)$. One can define 

\begin{equation}
    H \subset SO^+(3,1) = \left\{\Lambda^{\mu}_{\ \nu}p^{\nu} = p^{\mu}\right\}
\end{equation}
this is the \textit{little group} or the \textit{stabalizer}. It is the set of group elements that leave the momentum of the particle unchanged.
In this case $H = SO(3)$ which is the group of rotations in $\mathbb{R}^3$. Thus for the scalar rep: $S(\Lambda,0)\ket{p} = \ket{p}$ if $\Lambda\in SO(3)$. This means we are free to consider an $SO(3)$ irrep for each $\ket{p}$; or more generally $SU(2)$ (we showed that odd-dimensional irreps of $SU(2)$ are irreps of $SO(3)$).

\begin{equation}
    \mathcal{H}_{p,s} = \mathcal{H}_p\otimes\mathcal{H}_s
\end{equation}
where $\mathcal{H}_s = SU(2)$ irrep space of dimension $2s+1$ (we use the fact that since $SU(2)$ is compact its complex irreps are equivalent to unitary reps). Recall that $SU(2)$ irreps are labelled by dimension $2s+1$

\begin{itemize}
    \item $s = 0$: singlet
    \item $s = \frac{1}{2}$: doublet $v^{i} \ : \  \ket{\frac{1}{2}} =\begin{pmatrix}1\\0\end{pmatrix}$ and $\ket{-\frac{1}{2}}\begin{pmatrix}0\\1\end{pmatrix}$
    \item $s = 1$: triplet $v^{(ijk)} \ : \  \ket{1} =\begin{pmatrix}1 & 0\\0& 0\end{pmatrix}, \ \ \ \ket{0} = \begin{pmatrix}0 & 1\\1& 0\end{pmatrix}$ and $ \ket{-1} =\begin{pmatrix}0 & 0\\0& 1\end{pmatrix}$
    \item and so on.
\end{itemize}
Then for the irreps of Poincar\'e: 
\begin{equation}
    m^2 > 0, \text{ spins }\ \ \ \mathcal{H} = \bigoplus_{p^{\mu}: \ p^2 = m^2}\mathcal{H}_{p^{\mu},s}
\end{equation}

We really need to give,
\begin{equation}
    S(\Lambda, a)\ket{p^{\mu}, m} = ?
\end{equation}
Choose a \textit{standard boost}
\begin{equation}
    p^{\mu} = L^{\mu}_{\ \nu}(p)p_{0}^{\ \nu}
\end{equation}
where $p_{0}^{\ \nu} = (m,0,0,0)$. This is not unique. Given a general $\Lambda$ then, 
\begin{equation}
    L(\Lambda p)^{-1}\Lambda\cdot L(p)\in H(p_0)
\end{equation}
where $H(p_0)$ is the little group for $p_0$. We define 
\begin{equation}
    S(\Lambda, a)\ket{p^{\mu}, m} = e^{-i\Lambda p\cdot  a}\rho_s(W(\Lambda,p))^{m'}_{\ \ \ m}\ket{\Lambda,m'}
\end{equation}
where $\rho_s(W(\Lambda,p))^{m'}_{\ \ \ m}$ is the spin-$s$ rep. Next we embed the spin-$s$ rep into the \textit{non-unitary} $SO^{+}(3,1)$ rep. However, in each case 
\begin{itemize}
    \item \textit{spin-s} Hilbert space $\simeq{}$ \textit{solutions of wave-equation}
    \begin{equation}
    \begin{matrix}
    s=0 & \textit{Klein-Gordon equation } & (\Box + m^2)\psi(x)\\
    s=1/2 & \textit{Dirac equation} & (i\slashed{\partial}-m)\psi(x)\\
    s=1 & \textit{ Proca equation } & \partial^{\mu}(\partial_{\mu}A_{\nu} - \partial_{\nu}A_{\mu}) + m^2A_{\nu}=0\\
    \vdots & \vdots & \vdots
    \end{matrix}
    \end{equation}
\end{itemize}
Take the \textit{Proca equation}:
\begin{equation}
    \partial^{\mu}(\partial_{\mu}A_{\nu} - \partial_{\nu}A_{\mu}) + m^2A_{\nu}=0
\end{equation}
the solution $A_{\mu} = \varepsilon_{\mu}e^{-ip\cdot x}$ with $p^2 = m^2$ and $p^{\mu}\varepsilon_{\mu}=0$. The condition on the polarization vector shows 
\begin{equation}
    p^{\mu} = \begin{pmatrix}
        E,& 0,& 0, & 0
    \end{pmatrix} \text{ and } \varepsilon_{\mu} = \begin{pmatrix}
        0,& \varepsilon_x,& \varepsilon_y,&\varepsilon_z
    \end{pmatrix}
\end{equation}

Next we need to consider $m^2=0$. We can have,
\begin{equation}
    p^{\mu} = \begin{pmatrix}
        E & E& 0 &0
    \end{pmatrix}^{\mu}
\end{equation}
then the little group has the explicit form,
\begin{equation}
    \Lambda^{\mu}_{\ \ \nu} = \begin{pmatrix}
       1 + \frac{1}{2}(a^2+b^2) & -\frac{1}{2}(a^2+b^2)  &a&b \\
        \frac{1}{2}(a^2+b^2) & 1-\frac{1}{2}(a^2+b^2)  & a & b \\
       a\cos\theta + b\sin\theta & -(a\cos\theta + b\sin\theta)  & 1 & 0 \\
       -a\sin\theta + b\cos\theta  & a\sin\theta -b\cos\theta & 0 & 1
    \end{pmatrix}
\end{equation}
This can be split up in an interesting way, 
\begin{equation}
    \begin{split}
        \begin{pmatrix}
       1 &   &  & \\
         & 1 &  &  \\
         &   & \cos\theta & \sin\theta \\
         &   & -\sin\theta & \cos\theta
    \end{pmatrix}
    \begin{pmatrix}1 + \frac{1}{2}(a^2+b^2) & -\frac{1}{2}(a^2+b^2)  & a & b\\
        \frac{1}{2}(a^2+b^2) & 1-\frac{1}{2}(a^2+b^2)  & a & b \\
       a& -a  & 1 & 0 \\
       b & -b& 0 & 1
    \end{pmatrix}
    \end{split}
\end{equation}
What is this? Take a vector defined by
\begin{equation}
    v^{\mu} = \begin{pmatrix}
        1 + \frac{1}{2}(x^2+y^2)\\
        \frac{1}{2}(x^2+y^2)\\
        x\\
        y
    \end{pmatrix}
\end{equation}
then acting $\Lambda$ on $v$ (defining $r^2 = \frac{1}{2}(x^2 + y^2)$)
\begin{equation}
    \begin{split}
    &\begin{pmatrix}
       1 &   &  & \\
         & 1 &  &  \\
         &   & \cos\theta & \sin\theta \\
         &   & -\sin\theta & \cos\theta
    \end{pmatrix}
    \begin{pmatrix}1 + \frac{1}{2}(a^2+b^2) & -\frac{1}{2}(a^2+b^2)  & a & b\\
        \frac{1}{2}(a^2+b^2) & 1-\frac{1}{2}(a^2+b^2)  & a & b \\
       a& -a  & 1 & 0 \\
       b & -b& 0 & 1
    \end{pmatrix}
    \begin{pmatrix}
        1 + r^2\\
        r^2\\
        x\\
        y
    \end{pmatrix}\\
    &=\begin{pmatrix}1+\frac{1}{2}(x+a)^2 + \frac{1}{2}(y+b)^2\\
     \frac{1}{2}(x+a)^2 + \frac{1}{2}(y+b)^2\\
     (x+a)\cos\theta + (y+b)\sin\theta\\
     -(x+a)\sin\theta + (y+b)\cos\theta
    \end{pmatrix}
    \end{split}
\end{equation}
which is just 
\begin{equation}
    \begin{pmatrix}
        x\\y
    \end{pmatrix} \mapsto \begin{pmatrix}
        \cos\theta & \sin\theta\\
        -\sin\theta& \cos\theta
    \end{pmatrix}\begin{pmatrix}
        x\\y
    \end{pmatrix} = \begin{pmatrix}
        a\\b
    \end{pmatrix}
\end{equation}
Thus the little group is just $ISO(2)$ which is the group of symmetries on $\mathbb{R}^2$. The analysis is just as for $ISO^+(3)$: first consider \textbf{translations},
\begin{itemize}
    \item $\rho_{(k)} = e^{-i\vec{k}\cdot\vec{a}}$
    \item The little group for $\begin{pmatrix}
        k_1\\k_2
    \end{pmatrix}$ is \textit{trivial} so finished.
\end{itemize}
Unless $k=0$,
\begin{itemize}
    \item $\rho_{(\alpha)} = e^{-i\alpha\theta}$
\end{itemize}
We start by considering translations, with $\vec{a} = (a,b)$ then irreps are $1$-dimensional labelled by choice $\vec{k} = (k_1,k_2)$
\begin{itemize}
    \item $\rho_k = e^{-i\vec{k}\cdot\vec{a}}$
    \item Hilbert space $\mathcal{H}_k:$ $M(0,\vec{a})\ket{\vec{k}} = e^{-i\vec{k}\cdot\vec{a}}\ket{\vec{k}}$
\end{itemize}

For the full \textit{scalar} rep of $ISO(2)$: 
\begin{itemize}
    \item $\mathcal{H} = \bigoplus_{\vec{k}; \ k^2=t^2}\mathcal{H}_{\vec{k}}$
    \item $M(\theta,\vec{a})\ket{\vec{a}} = e^{-iR(\theta)\vec{k}\cdot\vec{a}}\ket{R(\theta)\vec{k}}$
\end{itemize}
What about the \textit{little group}? Stabilizer of $\vec{k}$ is $\mathbb{I}$ (trivial little group). However if we take the trivial rep: $\vec{k}=0$, then the stabilizer of $\vec{k}=0$ is $SO(2)$. So we have a family of irreps (not faithful) 
\begin{equation}
    M_{\alpha}(\theta,\vec{a}) = e^{-i\alpha\theta}; \ \ \ \alpha\in\frac{1}{2}\mathbb{Z}
\end{equation}
we allow reps when $\rho_{\alpha}(2\pi, \vec{a}) = -\rho_{\alpha}(2\pi, \vec{a})$. So there are two possibilities: 
\begin{enumerate}
    \item continuous spin labelled by $t^2$,
    \begin{equation}
        \mathcal{H} = \bigoplus_{\vec{k};k^2=t^2}\mathcal{H}_{\vec{k}}\ \ \ \ M(\theta,\vec{a})\ket{\vec{a}} = e^{-iR(\theta)\vec{k}\cdot\vec{a}}\ket{R(\theta)\vec{k}}
    \end{equation}
    \item \textit{helicity} labelled by $\alpha\in \frac{1}{2}\mathbb{Z}$
    \begin{equation}
        \text{\textit{1-dimensional,} } \mathcal{H}_{\alpha}\ \ \ \ \rho_{\alpha}(\theta, \vec{0}) = e^{-i\alpha\theta} 
    \end{equation}
\end{enumerate}
Physically, we do not see the continuous spin reps, so we define
\begin{equation}
    \mathcal{H}_{p,\alpha} = \mathcal{H}_p\otimes\mathcal{H}_\alpha
\end{equation}
Then the Poincar\'e irrep has, 
\begin{itemize}
    \item $m^2=0$, with helicity, $\alpha$
    \item $\mathcal{H} = \oplus_{p^{\mu};p^2=0}\mathcal{H}_{p_{\alpha}}$
    \item $S(\Lambda, a)\ket{p^{\mu}, \alpha} = e^{-i\Lambda p\cdot a}\rho_{\alpha}(W(\Lambda, p))\ket{\Lambda p, \alpha}$
\end{itemize}
Usually, one combines $\pm\alpha$ as $\pm 1$ helicity reps of spin $|\alpha|$. Again $\mathcal{H}$ can be viewed as the space of solutions of a \textit{massless equation}.

\section{Quarks, leptons and $SU(3)\times SU(2)\times U(1)$}

Let's now summaries the particles of the Standard Model and the \textit{gauge symmetries}. Recall that the \textit{matter} particles are all spin-$\frac{1}{2}$. We have 3 families of leptons with masses: 
\begin{equation}
    \begin{matrix}
        e^-\ (0.5 MeV)&\mu^-\ (106 MeV)&\tau^-\ (1.8 GeV)\\
        \nu_e\ (0.<10eV)&\nu_{\mu}\ (<0.16 MeV)&\nu_{\tau}\ (<18 MeV)& \text{ \textit{neutrinos}}\\
    \end{matrix}
\end{equation}

and 3 families of quarks: 
\begin{equation}
    \begin{matrix}
        u\ (\approx 0.4 GeV)& c\ (1.6 GeV) & t\ (175 GeV)\\
        d \ (\approx 0.4 GeV)&s\ (0.5 GeV)&b \ (4.5 GeV)&
    \end{matrix}
\end{equation}
We cannot observe quarks as free particles (\textit{confinement}) but only as bound stated (\textit{the hadrons}). The masses are \textit{effective} - the rough contribution of each quark to the bound state.

In SM mass for quarks and lepton comes from \textit{spontaneous symmetry breaking} via the Higgs particle. So to describe symmetries let's start by considering very high energies (\textit{electroweak scale}) where all fermions are effectively massless. We see that 
\begin{equation}
    \begin{matrix}
        m^2 = 0 & \text{spin-}\frac{1}{2} & \text{2 different helicity reps}\\
        &\mathcal{H}_{-\frac{1}{2}} & \mathcal{H}_{+\frac{1}{2}}\\
        & \textit{left-handed helicity}& \textit{right-handed helicity} 
    \end{matrix}
\end{equation}
so we can decompose for each particle into two reps, 

\begin{equation}
\begin{matrix}
    l^-\rightarrow l^-_{L},\ l^-_{R} & \text{and} & q\rightarrow q_{L},\ q_{R}
\end{matrix}
\end{equation}
where $l^-$ represents \textit{leptons} and $q$ represents the \textit{quarks}. Focusing on the \textit{Electroweak symmtry} via Salam and Weinberg. 
\begin{equation}
    \begin{matrix}
        \textit{electroweak}: & U(1)_{Y}\times SU(2)
    \end{matrix}
\end{equation}
we know the irreps 
\begin{equation}
    \begin{matrix}
        U(1)_{Y}: & \rho_{q} = e^{iq\theta}\\
        SU(2): & \underline{1}, \underline{2}, \underline{3}, \dots
    \end{matrix}
\end{equation}
So, how do the quarks and leptons form representations of $U(1)_{Y}\times SU(2)$?

Let's start with $SU(2)$, we have only \textit{doublets} and \textit{singlets} but the two different helicities are treated differently: 
\begin{itemize}
    \item doublets: $\underline{2}$
    \begin{equation}
        \begin{matrix}
            \begin{pmatrix}
                \nu_{e,L}\\
                e^{-}_{L}
            \end{pmatrix}, &
            \begin{pmatrix}
                \nu_{\mu,L}\\
                \mu^{-}_{L}
            \end{pmatrix}, &
            \begin{pmatrix}
                \nu_{\tau,L}\\
                \tau^{-}_{L}
            \end{pmatrix}\\ \\
            \begin{pmatrix}
                u_L\\
                d_L
            \end{pmatrix}, &
            \begin{pmatrix}
                c_L\\
                s_L
            \end{pmatrix}, &
            \begin{pmatrix}
                t_L\\
                b_L
            \end{pmatrix}\\
        \end{matrix}
    \end{equation}
    \item singlets: $\underline{1}$
    \begin{equation}
        \begin{matrix}
            e^-_R & \mu^-_R& \tau^-_R\\
            u_R, d_R& c_R, s_R & t_R, b_R
        \end{matrix}
    \end{equation}
\end{itemize}

The usual assumption is that $\nu_{e,R}$, $\nu_{\mu,R}$, $\nu_{\tau,R}$ which means that right-handed neutrinos don't get mass from the Higgs. Thus as fields we could write a \textit{doublet} of the Dirac fields or singlet

\begin{equation}
\begin{split}
    &\begin{matrix}
    l^i_L = \begin{pmatrix}
        \psi_{\nu_e,L}\\
        \psi_{e,L}
    \end{pmatrix}, & l_R = \psi_{e,R}
    \end{matrix}\\
&\begin{matrix}
    \mathcal{H} = &\mathcal{H}(m^2=0, \alpha=-\frac{1}{2})&\otimes&\mathcal{H}_{+\frac{1}{2}}\\
     & \text{Poincar\'e irrep space}& &SU(2) \text{ doublet space}
\end{matrix}
    \end{split}
\end{equation}
Next we need $U(1)_Y$, writing  $\rho_Y = e^{3iqY}$, where the $3$ is the conventional \textit{weak hypercharge} $Y = \{0\pm, \pm 1/3, \pm2/3, \dots \}$. We have
\begin{equation}
    \begin{matrix}
        \begin{pmatrix}
            \nu_{e,L}\\
            e^-_L
        \end{pmatrix}
        &
        \begin{pmatrix}
            \nu_{e,L}\\
            e^-_L
        \end{pmatrix} 
        & 
        \begin{pmatrix}
            \nu_{e,L}\\
            e^-_L
        \end{pmatrix}        
        &
        \ \ \ Y = -1\\ \\
        e^-_R
        &
        \mu^-_R
        &
        \tau^-_R
        &
        \ \ \ Y = -2\\ \\
         \begin{pmatrix}
            u_L\\
            d_L
        \end{pmatrix}
        &
        \begin{pmatrix}
            c_L\\
            s_L
        \end{pmatrix} 
        & 
        \begin{pmatrix}
            t_L\\
            b_L
        \end{pmatrix}        
        &
        \ \ \ Y = 1/3\\ \\
        u_R
        &
        c_R
        &
        t_R
        &
        \ \ \ Y = -4/3\\ \\
        d_R
        &
        s_R
        &
        b_R
        &
        \ \ \ Y = -2/3\\ \\
    \end{matrix}
\end{equation}
For each particle we can define the \textit{conjugate} \textit{anti-particle} by taking the \textit{conjugate representation}. Remember $\overline{\underline{2}}\sim{}\underline{2}$ for $SU(2)$ and for helicity $L\longleftrightarrow R$ under conjugation, so 
\begin{equation}
    \begin{matrix}
        e^+_{R} & \textit{is conjugate of }e^-_L & (+\longleftrightarrow - \textit{ for leptons})\\
        \overline{u}_{R} &  \textit{is conjugate of }u_L & (u, \overline{u} \textit{ for neutrinos and quarks})\\
    \end{matrix}
\end{equation}
So 
\begin{equation}
    \begin{matrix}
        \begin{pmatrix}
            \overline{e}^+_R\\
            \overline{\nu}_{e,R}
        \end{pmatrix}& \textit{doublet}& Y=+1& \textit{\dots for other left-handed leptons}\\ \\
        \overline{e}^+_L & \textit{singlet} & Y = +2 & \textit{\dots for other right-handed leptons}\\ \\
        \begin{pmatrix}
            \overline{d}_R\\
            \overline{u}_{R}
        \end{pmatrix}& \textit{doublet}& Y=-1/3& \textit{\dots for other right-handed quarks}\\ \\
        \begin{matrix}
            \overline{d}_R\\
            \overline{u}_{R}
        \end{matrix}& \textit{singlets}& \begin{matrix}
            Y=-4/3\\
            Y = 2/3
        \end{matrix}    
        & \textit{\dots for other left-handed quarks}\\ \\
    \end{matrix}
\end{equation}
Finally we also have,

\begin{equation}
\begin{matrix}
    SU(3) & \textit{colour symmetry} & \textit{quantum chromodynamics}
\end{matrix}
\end{equation}
We know that 
\begin{itemize}
    \item trivial representation $\underline{1}$
    \item defining representation $\underline{3}$
    \item conjugate of defining representation $\underline{\overline{3}}$.
\end{itemize}
we have 
\begin{itemize}
    \item all leptons are \textit{singlets}
    \item all quarks are \textit{triplets}, $\underline{3}$ (antiquarks are $\underline{\overline{3}}$)
\end{itemize}
so we note that each quark comes in 3 different types
\begin{equation}
\begin{matrix}
        u_R = \begin{pmatrix}
        u_{R,1}\\
        u_{R,2}\\
        u_{R,3}
    \end{pmatrix} & \begin{matrix}
        \longleftarrow\\
        \longleftarrow\\
        \longleftarrow
    \end{matrix} &
    \begin{matrix}
        \textit{red}\\
        \textit{green}\\
        \textit{blue}
    \end{matrix}
\end{matrix}    
\end{equation}
So for example $\begin{pmatrix}u_L\\d_L\end{pmatrix}$ has a representation space, 
\begin{equation}
\begin{matrix}
    \mathcal{H} = & \mathcal{H}(m^2=0,\alpha=-1/2)&\otimes&\mathcal{H}_{1/2}&\otimes&\mathcal{H}_{4/3}&\otimes&\mathcal{H}_{\underline{3}}\\
     & \uparrow& &\uparrow& &\uparrow& &\uparrow\\
     & \textit{Poincar\'e}& &SU(2)& &U(1)& &SU(3)\\
\end{matrix}
\end{equation}
\begin{equation}
    \begin{matrix}
        \psi= \psi^{ia}_{L} & i:\ \ SU(2) \textit{ doublet} & a: \ \ SU(3) \textit{ triplet}\\
        \textit{field}
    \end{matrix}
\end{equation}
